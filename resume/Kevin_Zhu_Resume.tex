% !TeX program = lualatex

\documentclass[11pt,letterpaper]{article}

\usepackage[margin=0.5in]{geometry}
\usepackage{enumitem}
\usepackage{fontspec}
\usepackage{xcolor}
%\setmainfont{Baskerville 10 Pro}
\setmainfont{Garamond ATF Text}
\setlength{\parindent}{0pt}
\setlist[itemize]{leftmargin=*,nosep}
\pagenumbering{gobble}
\newcommand{\sectline}{\vspace{5pt}\hrule height 1.5pt\vspace{5pt}}
\newcommand{\sectspace}{\vspace{10pt}}
\newcommand{\smallspace}{\vspace{5pt}}


\begin{document}
%%%% Name and Address %%%
\centering
{\fontsize{13pt}{13pt}\selectfont \textbf{KEVIN ZHU}} \vspace{2pt} \\
%{\fontsize{11pt}{13pt}\selectfont 370 Carmelita Place, Fremont, CA 94539 | kevinzhu98@gmail.com | (510) 706-8515}
{\fontsize{11pt}{13pt}\selectfont 370 Carmelita Place, Fremont, CA 94539 | kevin.zhu01@libertymutual.com | (510) 706-8515}
\sectspace

\fontsize{11pt}{13pt}\selectfont
%%% Actuarial Exams %%%
\raggedright
\textbf{CAS EXAMS}\sectline
\begin{itemize}
	\item Passed Exams 1, 2, 3F, 5, MAS-I, OC1; Completed VEE Economics, VEE Accounting \& Finance
	\item Sitting for Exam 6-US (\textit{May 2021})
\end{itemize}
\sectspace

%%% Work Experience %%%
\textbf{WORK EXPERIENCE}\sectline
\textbf{Liberty Mutual Insurance}  \hfill Seattle, WA \\
\textit{Actuarial Assistant, US Business Lines} \hfill \textit{Jul 2019 -- Present}
\begin{itemize}
	\item Created database containing granular on-leveling, trending, developing, smoothing, and credibility weighting of BOP losses and premiums for use in rate indications and various ad-hoc analyses
	\item Developed and implemented logic for on-leveling policies from different rating platforms onto a single uniform basis, including consideration for on-leveling on minimum premium and mods
	\item Identified inconsistencies and errors in new underlying data source, back-filling from previous source and relying on parallelogram estimation techniques as necessary to produce accurate on-leveled premiums
	\item Estimated swing between prior and current indication, reconciling actual and expected rate changes
\end{itemize}
\smallspace
\textbf{Milliman, Inc.} --- Property and casualty actuarial consulting\hfill San Francisco, CA \\
\textit{Actuarial Intern} \hfill \textit{Jun 2019 -- Sep 2019}
\begin{itemize}
	\item Estimated IBNER and pure IBNR hurricane losses for a homeowners' insurer using a frequency-severity technique, varying trend assumptions and selected tail factors to create a range of reasonable estimates
	\item Conducted an actual versus expected loss and ALAE emergence analysis, reconciled updated data to prior reserve reviews, and investigated accident quarters with higher-than-expected emergence
%	\item Resolved inconsistencies and data quality issues between several decades of flood insurance policy and claim data, determining the optimal set of variables on which to join to minimize missing and duplicate records 
	\item Simulated a market basket of homeowners' policies and built a rater in SAS to compare average premium by rating variable across competitors, creating accompanying exhibits for the Department of Insurance
\end{itemize}
\smallspace 
\textbf{Capital Insurance Group} --- Property and casualty insurance company \hfill Monterey, CA \\
\textit{Actuarial Intern} \hfill \textit{Jun 2018 -- Sep 2018}
\begin{itemize}
	\item Completed quarterly dwelling fire rate indication and inland marine reserve review, using actuarial judgment to select trends, development factors, ultimate losses, and indicated rate changes
	\item Built a GLM in Python to model homeowners pure premium, conducting cost-benefit analyses on installing water loss prevention devices in selected subsets of high-risk homes 
	\item Designed a user-friendly businessowners policy renewal tool in Tableau for underwriters and management to easily retrieve specific policy details and summaries of segmented data	
%	\item Identified key drivers of an increase in homeowners claim severity by investigating causes of loss
\end{itemize}
%\smallspace 
%\textbf{UCLA Mathematics Computing Lab}\hfill Los Angeles, CA \\
%\textit{Head PIC Lab Assistant} \hfill \textit{Sep 2018 -- Jun 2020} \\
%\begin{itemize}
%	\item Supervised and trained lab assistants, disseminating information from the math department when necessary
%	\item Guided students through C++ programming assignments, helping with basic concepts and debugging
%	\item Fixed basic hardware, software, and network problems with lab equipment
%\end{itemize}
\sectspace

%%% Education %%%
\textbf{EDUCATION}\sectline
\textbf{University of California, Los Angeles} \hfill Los Angeles, CA \\
\textit{B.S. Mathematics/Economics; Specialization in Computing; Minor in Accounting} \hfill \textit{Sep 2016 -- Jun 2020}
\begin{itemize}
	\item Graduated \textit{summa cum laude} (GPA: 4.00/4.00), Elected Phi Beta Kappa, Received Outstanding Mathematics / Economics Student Award (given to five  top-ranked students based on faculty recommendations)
	\item Competed in the California Actuarial League Case Competition, winning Best Solution in the Health and Benefits and Property and Casualty tracks (\textit{2018}) and Best Individual Presenter in the Retirement track (\textit{2017})
\end{itemize}
\sectspace

%%% Leadership %%%
\textbf{LEADERSHIP} \sectline
\textbf{Bruin Actuarial Society} \hfill Los Angeles, CA \\
\textit{President} \hfill \textit{May 2019 -- May 2020}
%\textit{Director of Professional Development} \hfill \textit{May 2018 -- May 2019} \\
%\textit{Corporate Liaison} \hfill \textit{May 2017 -- May 2018}
\begin{itemize}
%	\item Served previously as Director of Professional Development (\textit{2018 -- 2019}) and Corporate Liaison (\textit{2017 -- 2018})
	\item Led a team of 6 officers and corresponded with actuaries from various firms to plan and execute an annual career fair, case competition, banquet, and various workshops for hundreds of members
%	\item Wrote 5 technical workshops incorporating simulated data to introduce Excel and SQL in actuarial contexts
	\item Designed and presented a new series of workshops about P\&C insurance, introducing members to ratemaking and reserving concepts by walking through examples in Excel
\end{itemize}
\sectspace

%%% Awards %%%
%\textbf{AWARDS} \sectline
%\textbf{California Actuarial League Case Competition} \hfill Berkeley, CA \\
%\begin{itemize}
%\item Best Solution (Health \& Benefits, P\&C Tracks) \hfill \textit{Feb 2018 -- Apr 2018} \\
%\item Finalist (P\&C and Retirement Tracks); Best Individual Presenter (Retirement Track) \hfill \textit{Mar 2017 -- Apr 2017}
%\end{itemize}
%\begin{itemize}
%	\item Designed standalone health insurance plans, analyzed the effects of offering them as multi-choice options, and mitigated adverse selection risk by simulating enrollment and adjusting premiums
%	\item Mitigated adverse selection risk in offering standalone health insurance plans as multi-choice options
%	\item Modified homeowners' insurance pricing factors by analyzing rate adequacy by segment, minimizing policy-level premium dislocation while ensuring sufficient increase in total premium
%	\item Adjusted homeowners' insurance rate relativities, minimizing policy-level premium dislocation 
%	\item Evaluated an individual's retirement adequacy under the final average pay, cash balance, and 401(k) plans, performing sensitivity analysis on our qualitative and quantitative assumptions
%\end{itemize}
%\smallspace 
%\begin{itemize}
%	\item Calculated the Medicaid bid and Medicaid Rebate based on historical data to price a health plan
%	\item Analyzed reinsurance plans for a homeowners' insurance firm, computing TVaR for simulated losses
%	\item Recommended risk-reducing actions for a DB plan sponsor, considering lump sums and buyouts 
%\end{itemize}

%\textbf{Bruin Actuarial Society Molina Healthcare Fifth Annual Case Competition} \hfill Los Angeles, CA \\ 
%\begin{itemize}
%	\item Projected claims PMPM using historical claims data, considering potential changes in membership, region, and demographics as well as applying historical utilization, unit cost, and claims trends
%	\item Adjusted our model based on area factors, demographic adjustments, and age calibration
%	\item Priced the healthcare plan, offering a solution that would retain the historical profit margin 
%\end{itemize}
%\sectspace

%%% Additional %%%
\textbf{ADDITIONAL}\sectline
\begin{itemize}
	\item \textbf{Computer Languages:} Python, SAS, R, SQL, VBA, Teradata
%	HTML \& CSS, JavaScript, PHP, C++, \LaTeX
	\item \textbf{Other Tools}: Advanced Microsoft Excel, Intermediate Tableau, Introductory Microsoft Access
%	\item \textbf{Languages:} Conversational Mandarin Chinese
\end{itemize}

\end{document}