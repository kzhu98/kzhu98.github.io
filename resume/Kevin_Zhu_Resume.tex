%!TeX program = lualatex

\documentclass[11pt,letterpaper]{article}

\usepackage[margin=0.5in]{geometry}
\usepackage{enumitem}
\usepackage{fontspec}
\usepackage{xcolor}
%\setmainfont{Baskerville 10 Pro}
\setmainfont{Garamond ATF Text}
\setlength{\parindent}{0pt}
\setlist[itemize]{leftmargin=*,nosep}
\pagenumbering{gobble}
\newcommand{\sectline}{\vspace{5pt}\hrule height 1.5pt\vspace{5pt}}
\newcommand{\sectspace}{\vspace{9pt}}
\newcommand{\smallspace}{\vspace{6pt}}
\newcommand{\red}[1]{{\color{red}#1}}
\newcommand{\heading}[1]{{\fontsize{12pt}{13pt} {\textbf{\textsc{#1}}}}}


\begin{document}
%%%% Name and Address %%%
\centering
{\fontsize{13pt}{13pt}\selectfont \textbf{KEVIN ZHU,\hspace{1.7 mm}FCAS}} \vspace{2pt} \\

{\fontsize{11pt}{13pt}\selectfont Brooklyn, NY | kevinzhu98@gmail.com | (510) 706-8515}
\sectspace

\fontsize{11pt}{13.7pt}\selectfont
\raggedright

%%% Work Experience %%%
\heading{Work Experience}\sectline
\textbf{Farmers Insurance} \hfill Woodland Hills, CA \textit{(Remote)}\\
\textbf{\textit{Associate Actuary FCAS, Business Insurance}} \hfill \textit{Aug 2023 -- Present}
\begin{itemize}
	\item Led the development of proprietary rating models for a new BOP product, collaborating with stakeholders in underwriting, product, and data science to refine assumptions, scope, and appropriate feature engineering
	\item Trained and mentored two new team members and one direct report; peer reviewed and provided constructive feedback on contributions to data extraction, actuarial indication, and predictive modeling processes
	\item Built a new rate indication process for a transportation network company accounting for \$1B in gross premium, formulated internal bidding strategy, and led rate negotiations with the insured's actuarial team
	\item Modeled pure premium using a GLM in Python to update building age curves 
\end{itemize}
\textbf{\textit{Assistant Actuary ACAS, Business Insurance}} \hfill \textit{Aug 2022 -- Aug 2023}
\begin{itemize}
	\item Designed a policy/claim level BOP database for use in indications and rate revisions, including allocation of ultimate losses and claim counts by peril using record-level Bornhuetter-Ferguson
	\item Supported data extracts and transformations (focusing on loss segmentation and development by peril) for a BOP pure premium XGBoost model and facilitated alignment with actuarial rate indications
	\item Prepared rate support exhibits for TNC and BOP rate filings and defended proposals against DOI objections
\end{itemize}
\smallspace
\textbf{Liberty Mutual Insurance}\hfill Seattle, WA \\
\textbf{\textit{Actuarial Analyst, Specialty Reserving}} \hfill \textit{Feb 2022 -- Aug 2022} \\
\begin{itemize}
	\item Realigned medical malpractice, D\&O, and E\&O reserve classes for more granular unpaid claim analyses by splitting aggregate data, correcting internal inconsistencies, and selecting new payment and reporting patterns
%	\item Corrected internal inconsistencies in claim count and loss data by leveraging alternative data sources and interpolating when data is unavailable, preparing reserving datasets for frequency/severity analyses
	\item Refined mappings between pricing segments and reserve classes for use in development of expected loss ratios, driving alignment of rate and trend assumptions across multiple teams
\end{itemize}
\textbf{\textit{Actuarial Assistant, Business Lines Indications}} \hfill \textit{Jul 2020 -- Feb 2022}
\begin{itemize}
	\item Created database containing granular on-leveling, trending, developing, large loss smoothing, and credibility weighting of BOP losses and premiums in SAS for rate indications and various ad-hoc analyses
	\item Automated procedure to quantify effects of changes in loss experience, rate level, LDF/trend selections, CAT load, and expense/profit provisions on state indications, reducing refresh time from 1 week to 15 minutes
	\item Compared various external benchmarks to quantify COVID impact on BOP liability losses, estimating state-level adjustments to indications to produce prospective estimates for post-COVID loss periods
%	\item Developed logic for on-leveling policies from different rating platforms onto a single uniform basis, including consideration for minimum premium, premium modification factors, missing rates, and inaccurate data
	\item Evaluated rate need for a new BOP product by leveraging profitability insights on similar existing GL products, adjusting data for rate level and book-mix differences to produce credible state-level comparisons
\end{itemize}
\smallspace
\textbf{Milliman} \hfill San Francisco, CA \\
\textbf{\textit{Actuarial Intern}} \hfill \textit{Jun 2019 -- Sep 2019}
\begin{itemize}
%	\item Projected IBNER and pure IBNR hurricane losses for a homeowners insurer using a frequency-severity technique, varying trend assumptions and selected tail factors to create a range of reasonable estimates
	\item Projected unpaid hurricane losses for a homeowners insurer using frequency-severity techniques
%	\item Conducted an actual versus expected loss and ALAE emergence analysis, reconciled updated data to prior reserve reviews, and investigated accident quarters with higher-than-expected emergence
%	\item Resolved inconsistencies and data quality issues between several decades of flood insurance policy and claim data, determining the optimal set of variables on which to join to minimize missing and duplicate records 
%	\item Simulated a market basket of homeowners policies and built a rater in SAS to compare average premium by rating variable across competitors, creating accompanying exhibits for the Department of Insurance
	\item Simulated a market basket of homeowners policies to compare competitors' premium by rating variable
%	\item Conducted an AvE loss emergence analysis and investigated quarters with higher-than-expected emergence
\end{itemize}
\smallspace 
\textbf{Capital Insurance Group} \hfill Monterey, CA \\
\textbf{\textit{Actuarial Intern}} \hfill \textit{Jun 2018 -- Sep 2018}
\begin{itemize}
%	\item Completed quarterly dwelling fire rate indication and inland marine reserve review, using actuarial judgment to select trends, development factors, ultimate losses, and indicated rate changes
	\item Identified high-risk homes and conducted cost-benefit analyses on installing water loss prevention devices
	\item Built a BOP renewal tool in Tableau to display both specific policy details and summaries of segmented data	
	% 	\item Built a user-friendly BOP renewal tool in Tableau for underwriters and management to easily retrieve specific policy details and summaries of segmented data	
%	\item Identified key drivers of an increase in homeowners claim severity by investigating causes of loss
\end{itemize}
%\smallspace 
%\textbf{UCLA Mathematics Computing Lab}\hfill Los Angeles, CA \\
%\textit{Head PIC Lab Assistant} \hfill \textit{Sep 2018 -- Jun 2020} \\
%\begin{itemize}
%	\item Supervised and trained lab assistants, disseminating information from the math department when necessary
%	\item Guided students through C++ programming assignments, helping with basic concepts and debugging
%	\item Fixed basic hardware, software, and network problems with lab equipment
%\end{itemize}
\sectspace

%%% Education %%%
\heading{Education}\sectline
\textbf{University of California, Los Angeles} \hfill Los Angeles, CA \\
\textit{B.S. Mathematics/Economics; Specialization in Computing; Minor in Accounting} \hfill \textit{Sep 2016 -- Jun 2020}
\begin{itemize}
	\item Graduated \textit{summa cum laude} (GPA: 4.00/4.00), Elected Phi Beta Kappa%, Received Outstanding Mathematics / Economics Student Award (given to five  top-ranked students based on faculty recommendations)
	\item Served as President of Bruin Actuarial Society, UCLA's premier organization for student actuaries 
%	\item Competed in the California Actuarial League Case Competition, winning Best Solution in the Health and Benefits and Property and Casualty tracks (\textit{2018}) and Best Individual Presenter in the Retirement track (\textit{2017})
\end{itemize}
\sectspace

%%% Leadership %%%
%\textbf{LEADERSHIP} \sectline
%\textbf{Bruin Actuarial Society} \hfill Los Angeles, CA \\
%\textit{President} \hfill \textit{May 2019 -- May 2020}
%\textit{Director of Professional Development} \hfill \textit{May 2018 -- May 2019} \\
%\textit{Corporate Liaison} \hfill \textit{May 2017 -- May 2018}
%\begin{itemize}
%	\item Served previously as Director of Professional Development (\textit{2018 -- 2019}) and Corporate Liaison (\textit{2017 -- 2018})
%	\item Led a team of 6 officers and corresponded with actuaries from various firms to plan and execute an annual career fair, case competition, banquet, and various workshops for hundreds of members
%	\item Wrote 5 technical workshops incorporating simulated data to introduce Excel and SQL in actuarial contexts
%	\item Introduced new workshops incorporating Excel examples to illustrate P\&C pricing and reserving concepts
%\end{itemize}
%\sectspace

%%% Awards %%%
%\textbf{AWARDS} \sectline
%\textbf{California Actuarial League Case Competition} \hfill Berkeley, CA \\
%\begin{itemize}
%\item Best Solution (Health \& Benefits, P\&C Tracks) \hfill \textit{Feb 2018 -- Apr 2018} \\
%\item Finalist (P\&C and Retirement Tracks); Best Individual Presenter (Retirement Track) \hfill \textit{Mar 2017 -- Apr 2017}
%\end{itemize}
%\begin{itemize}
%	\item Designed standalone health insurance plans, analyzed the effects of offering them as multi-choice options, and mitigated adverse selection risk by simulating enrollment and adjusting premiums
%	\item Mitigated adverse selection risk in offering standalone health insurance plans as multi-choice options
%	\item Modified homeowners' insurance pricing factors by analyzing rate adequacy by segment, minimizing policy-level premium dislocation while ensuring sufficient increase in total premium
%	\item Adjusted homeowners' insurance rate relativities, minimizing policy-level premium dislocation 
%	\item Evaluated an individual's retirement adequacy under the final average pay, cash balance, and 401(k) plans, performing sensitivity analysis on our qualitative and quantitative assumptions
%\end{itemize}
%\smallspace 
%\begin{itemize}
%	\item Calculated the Medicaid bid and Medicaid Rebate based on historical data to price a health plan
%	\item Analyzed reinsurance plans for a homeowners' insurance firm, computing TVaR for simulated losses
%	\item Recommended risk-reducing actions for a DB plan sponsor, considering lump sums and buyouts 
%\end{itemize}

%\textbf{Bruin Actuarial Society Molina Healthcare Fifth Annual Case Competition} \hfill Los Angeles, CA \\ 
%\begin{itemize}
%	\item Projected claims PMPM using historical claims data, considering potential changes in membership, region, and demographics as well as applying historical utilization, unit cost, and claims trends
%	\item Adjusted our model based on area factors, demographic adjustments, and age calibration
%	\item Priced the healthcare plan, offering a solution that would retain the historical profit margin 
%\end{itemize}
%\sectspace

%%% Additional %%%
\heading{Additional}\sectline
\begin{itemize}
	\item \textbf{Computer Languages and Tools}: SQL, Python, SAS, Microsoft Excel, VBA, Power BI, Git, ResQ
%	HTML \& CSS, JavaScript, PHP, C++, \LaTeX, R
%	Intermediate Tableau
\end{itemize}

\end{document}